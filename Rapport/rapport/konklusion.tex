Vi har igennem arbejdet med projektet, form�et at samle en god del viden, samt
teknikker mht. konstruktion af prototyper, scenarier og hvordan man bedst muligt
inkludere brugerne i en udviklingsproces, s�ledes at et eventuelt slut produkt,
vil komme til, at opfylde brugerens krav og behov.

Vi har desuden f�et erfaring med iterativt design, og inddragelse af brugeren i
denne, samt en grundl�ggende forst�else for system udvikling, og bruger
interaktion.

Vi har udvidet vores evne, til opstarte et projekt p� egen h�nd, bl.a. gennem
virksomhedskontakt. Tilsidst har vi ogs� l�rt at reflektere over vores proces,
og produkt, hvilket kommer til udtryk igennem denne raport.

Vi vurdere at vi igennem kurset har opn�et st�rre fagligt indenfor dom�net,
hvilket har gjort os i stand til, at s�tte sp�rgsm�lstegn ved, og vurdere vores
samlede arbejdsproces igennem projektet, og dermed ogs� ligge m�rke til at vi
kunne have gjort nogle aspekter af vores proces om, for eventuelt at f� st�rre
udbytte af vores brugerinteraktion. Hvilket �benlyst kunne have f�rt projektet i
en anden retning.

Omvendt har vi ogs� v�ret i stand til at se de gode ting, vi har f�et ud af
vores brugerinteraktion, is�rdeles at brugerne, for hver interation blev mere og
mere positive, overfor prototypen, hvormed vi har kunnet konkludere at vi er
kommet n�rmere deres endelige �nske for produktets udformning.

Endelig har vi desv�rre m�ttet konkludere, at kursets l�betid, har v�ret yderst
begr�nsende, idet vi ikke har v�ret i stand til at komme igennem nok iterationer,
til faktisk at kunne f�rdigg�re et produkt, og se det blive taget i brug.

At se produktet blive taget i brug, og forts�tte designprocessen efter dette, i
form af vedligeholdelse og optimering, havde ogs� v�ret sp�ndende, omend en
smule uden for kursets dom�ne.

Vi konkludere samlet set, at vi har f�et et stort udbytte af kurset, og at vi
f�ler os bedre rustet i fremtiden, til at lave brugerdrevet udvikling, vha.
iterative fremgangsm�der.
