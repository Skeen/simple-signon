%TODO: Hvordan vi har arbejdet med brugerne?
%TODO: Hvad har vi f�et ud af at arbejde med brugerne?

Oprindeligt var vi lidt negativt stemt over tanken, med brugerinddragelse og
brugersamarbejdet, nok prim�rt, da vi havde h�rt at kurset var en form for
'dIntDesign 2.0'.

Men da vi var kommet i gang med projektet, og havde afholdt vores f�rste
interview med Aarhus Tech, og derigennem f�lt deres entusiasme for at f� l�st et
problem i deres arbejdsgang, blev vi langt mere motiveret for projektet.

Dette blev da ogs� kun forst�rket ved vores f�rste l�re interview, hvor vi fik
forklaret, hvor stort et problem vi faktisk havde at g�re med. - Og ikke mindst
hvor gerne, l�rene s� en l�sning p� problemet.

I dette f�rste interview med l�rene, fik vi da ogs� indblik i hvor megen
spildtid, der fulgte med at skulle logge ind p� de mange forskellige systemer,
samt fustrationerne der var forbundet med, at skulle huske de mange forskellige
logins.

Da vi herefter pr�senterede vores meget skrapede f�rste prototype, fik vi meget
respons, og en hel del af denne, faktisk respons, som vi fandt yderst brugbart
og hj�lpsom, ifht. hvilket retning vi skulle udvikle vores prototype.

Eksempelvis, kom den ene af l�rene med et decideret forslag til hvordan
brugergr�nse fladen kunne se ud, delvist inspireret af touchscreen interfaces,
og de store indbydende ikoner, der typisk findes her.

I de l�bende interviews, fik vi p� samme m�de positiv, og ikke mindst negativ
feedback, som har v�ret en ledesnor for vores udviklingsproces, igennem hele
projektet.

S� for at opsummere, har vi haft et n�rtsamarbejde med brugerne, hvor de har
fungeret som en central del af produkt udviklingen, og vi har dermed f�et
utroligt meget ud af at inddrage brugerne i udviklingen.
