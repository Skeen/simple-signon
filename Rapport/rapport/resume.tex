%Motivation (our)
As part of the 'Eksperimentel Systemudvikling', at Aarhus University,
we were tasked to do a project with an external organisation of our choosing.
%Problem Statement
This project was done in cooperation with Aarhus Tech, a education focused organisation, 
who asked us to develop a single-sign-on system to their many independant services, 
many of which use seperated usernames and passwords, and to make such a system suitable to its users.
%Approach
Using Contextual interviews, prototyping and iterative design, 
we have attempted to identify technical requirements, 
and then develop a solution over several iterations, each consisting of an interview, 
some development and questions we attempted to answer in the following iteration.
%Results 
The Simple-SignOn system is the result 
and nearly contains the functionality that we had attempted to implement.
It's design and functionality are a balanced compromise of time limit, functionality, 
user usability needs and the requirements posed by the course.
%Conclusion
The project and interviews has given us a better understanding of 
methods for analysing organisation structure, design methods and user interaction.
