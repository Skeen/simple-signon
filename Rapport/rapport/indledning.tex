%Her skal vi skrive en indledning til projektet.

I forbindelse med kurset Eksperimentel Systemudvikling, Er vi blevet stillet til
opgave, at finde en virksomhed, med hvem vi kunne designe et system efter deres
�nske.  Vi har indg�et et samarbejde med Aarhus Tech, om at lave et
Single-sign-on system, der har til form�l, at medhj�lpe et problem de har med
medarbejdere der skal huske mange forskellige mange brugernavne og kodeord til
mange forskellige services.  Vi har fors�gt at benytte metoderne, som blev
undervist i kurset, til at lave en l�sning, der passer til Aarhus Tech.  Vi har
arbejdet iterativt, hvor hver iteration har indeholdt en samtale med brugerne, i
form af kontekstuel interview, observation, prototyping eller flere p� en gang.
Derefter har vi fors�g at bruge det feedback, som brugerne har givet til at
forbedre vores design og tage de bedste beslutninger med hensyn til systemet.  I
denne rapport, giver vi f�rst en beskrivelse af Aarhus Tech, hvilke dele af
institutionen der har v�ret relevante i forbindelse med projektet og hvordan
disse spiller sammen.  Dette bliver suppleret af en PACT-analyse af Aarhus Tech
samt et senarie der beskriver brugen af det nuv�rende system.  Herefter har vi
en beskrivelse af hvilke metoder vi har brugt, og hvordan selve processen er
forl�bet.  Dern�st indg�r der en beskrivelse af de erfaringer vi har gjort os i
forhold til det brugersamarbejde vi har haft med Aarhus Tech.  Efterf�lgende
inddrager vi 2 scenarier der har til form�l at beskrive hvordan en bruger vil
komme til at benytte vores system.  Da Systemet ikke er f�rdig, er disse
scenarier er baseret p� vores nuv�rende prototyper.  Herefter diskuterer vi de
problemer vi har m�dt undervejs i projektet og de l�sninger vi har valgt.
Derudover diskuterer vi ogs� vores processtrategi og perspektivere vores design
og brug af prototyper til en artikel af Y.-K. Lim et al.  Til sidst i rapporten
konkludere vi p� hvad b�de dette projekt har givet os, men ogs� hvad faget
Eksperimentel Systemudvikling som helhed har givet os.  
