%TODO: l�s Tidslinje for vores arbejdsprocess
%TODO: detaljer for v�sentlige steder i tidslinjen.

Vi sendte emails ud til flere skoler, for at pr�ve at starte et samarbejde.
Lars Lisberg, IT-teamleder ved Aarhus Tech, skrev tilbage at de var interesserede i et samarbejde.

Vi forberedte et interview med Lars


{\bf Interview 1, 20/03}
med Lars Lisberg, IT-teamleder ved Aarhus Tech.


Baseret p� dette f�rste interview begyndte vi at lave designet til en prototype, og tale om hvilke krav der er til denne.

Da dette var t�t p� en eksamensperiode, blev arbejdet sat p� pause indtil efter eksamensperioden.
Vi m�dtes igen d. 21/04 og forstatte arbejdet. 

For at f� bekr�ftet Lars Lisbergs teori om at de mange logins skaber irritationer og problemer hos l�rer, kontaktede vi Aarhus tech igen om at f� et interview.

{\bf Interview 2, 23/04}
med Martin og Martin, to l�rer ved Aarhus Tech.

Her fik vi bekr�ftet at mange logins er et egentligt problem 

Iterativt udvikling af designet baseret p� bruger feedback fra interviewet.

Udvikling af funktionalit bag prototype designet

{\bf Interview 3, 07/05}
med Lars Lisberg, IT-teamleder ved Aarhus Tech.

Underlig sammenh�ng imellem Java klasser blev erstattet af et Event system. 
Denne lader diverse klasser lave et 'event' indvendigt, i stedet for at alle klasser skal have referencer til hinanden.


{\bf Interview 4, 24/05}
med Martin, martin og Karin, to l�rer og en uddannelses leder ved Aarhus Tech.

Til dette sidste interview blev der givet feedback og kritik til den 'endelige' prototype.

