%TODO: Beskriv via scenarie hvordan VORES prototype kommer til at blive brugt.
% Dette skal bruges til reflektion, og skal vise evt. fejl/mangler med den prototype vi afleverer.

Martin(en l�rer) m�der om morgenen p� Aarhus Tech. Da Martin i sidste uge var p� kursus i Kolding, skal han benytte Travel-X til at rapportere rejseomkostningerne forbundet med kurset. Da Travel-X ikke ligger p� hans arbejdscomputer, men i clouden, ville Martin dog f�rst skulle logge p� et tr�dl�st netv�rk, derefter logge p� VPN for til sidst at indtaste/huske sit brugernavn og kodeord til Travel-X. Alle disse indtastninger slipper Martin dog heldigvis for ved at benytte programmet SimpleSignon som ligger lokalt p� hans computer. Hans skal blot �bne programmet, �n gang indtaste sit brugernavn og adgangskode, og derefter kan han straks �bne Travel-X via et link i programmet. 
Da Martin ogs� st�r for at l�gge skema for de studerende, begynder SimpleSignon for alvor at spare ham tid, for i mods�tning til i gamle dage, hvor han nu skulle p� nettet for at tilg� Elevplan og her igen indtaste (forskellige fra de andre)brugernavn og adgangskode, kan han nu med SimpleSignon blot trykke p� linket der repr�senterer Elevplan, og straks er han inde p� siden logget ind og klar til at arbejde.