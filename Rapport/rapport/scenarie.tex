% TODO: Beskriv via scenarie hvordan VORES prototype kommer til at blive brugt.
% Dette skal bruges til reflektion, og skal vise evt. fejl/mangler med den prototype vi afleverer.

F�rste senarie er en historie der omhandler en l�rer p� Aarhus Tech ved navn Martin, og hvordan han benytter den prim�re id� og funktionalitet ved Simple-SignOn.

{\Large Senarie 1:}
Martin(en l�rer p� Aarhus Tech) m�der om morgenen p� Aarhus Tech.\\
Da Martin i sidste uge var p� kursus i Kolding, skal han benytte Travel-X til at rapportere rejseomkostningerne forbundet med kurset. 
Travel-X ligger dog ikke direkte p� Martins arbejdscomputer, men i clouden.
Dette havde betydet, at Martin f�r i tiden f�rst skulle logge p� et tr�dl�st netv�rk, derefter logge p� VPN for til sidst at indtaste/huske sit brugernavn og kodeord til Travel-X. 
Alle disse indtastninger slipper Martin dog heldigvis for idag.
Da Martin benytter programmet SimpleSignon som ligger lokalt p� hans computer, skal han skal blot �bne programmet, og �n gang indtaste sit brugernavn og adgangskode. 
Herefter kan han straks �bne Travel-X via et link i programmet og rapportere sine rejseomkostninger.\\ 
Da Martin ogs� st�r for at l�gge skema for de studerende, begynder SimpleSignon for alvor at spare ham tid. 
I mods�tning til i gamle dage, hvor han nu skulle p� nettet for at tilg� Elevplan og her igen indtaste (forskellige fra de andre)brugernavn og adgangskode, kan han nu, da han er logget p� SimpleSignon, blot trykke p� linket der repr�senterer Elevplan, og straks er han inde p� Elevplan, logget ind og klar til at arbejde.


\leavevmode \linebreak

Andet senarie omhandler en medarbejder p� Aarhus Tech ved navn Gitte, og hvordan hun, f�rste gang hun benytter programmet kommer igennem mange af funktionaliteterne.

{\Large Senarie 2:}


