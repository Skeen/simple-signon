% Brugt i samarbejde med brugere

{\bf High-fidelity prototyping}
% De var interaktive programmer, med faket funktionalitet
Efter vores f�rste interview, med Lars Lisberg (IT leder p� Aarhus Tech), valgte
vi at udvilke en meget skrabet high-fidelity prototype, idet vi vurdere at en
high-fidelity prototype ville give os bedre feedback, end en low fidelity
prototype. 

Vi var forsigtige med, at designe mere end vi havde f�et �nsket af Lars, idet vi
udelukkende havde en persons holdning til problemstillingen, den f�rste
prototype fik vi lavet f�rdig p� f� timer.

Da vi fremviste prototypen ved f�rste interview, var brugerne (de to l�rer,
Martin og Martin), positivt stemte, men gjorde os opm�rksom p� mange problemer,
og forslag til �ndringer af prototypen.

Som et specifikt eksempel, lagde vi m�rke til at den ene l�re instiktivt klikkede
{enter}, for at ville logge ind, og gik lidt i st�, da intet skete, p� denne
m�de havde vi fundet ud af en mangel i vores prototype, som kunne benyttes til at
udvikle prototypen i den retning, som brugerne benyttede den.

Vi fik desuden meget feedback, p� brugergr�nse fladen, som blev beskrevet som
'for teknisk', og med for meget (for brugeren) ligegyldig information, dette
f�rte os til at bede brugeren om, at komme med deres vision for prototypens
brugergr�nse flade, hvilket vi s� kunne implementere til den anden prototype.

P� denne m�de, igennem en iterativ fremgangsm�de, lavede vi flere iterationer af
prototypen, som hver gang kom n�rmere brugerens �nsker og behov.

{\bf Iterativt design}
% Er ogs� et krav i projekt beskrivelsen at prototype er udviklet over flere iterationer
Som n�vnt ovenfor, har vi arbejdet iterativt med brugeren, s�ledes at deres
feedback har direkte aff�dt �ndringer i vores high-fidelity prototype, igennem
flere iterationer, den iterative udvikling kan da ogs� ses i 
\hyperref[1Prototype]{appendixet}.

Projektet blev udviklet iterativt, i forhold til det feedback, som brugere 


{\bf Contextual interview}



{\bf User Experience Design}


% Internt brugt i gruppen

{\bf Low-fidelity prototyping}
% tegnede designet p� tavlen



{\bf Scenarie}
% Brugte scenarier da vi diskutterede design



{\bf Sequence model}



{\bf Use Case}
% Use case er en beskrivelse af en interaktion i mellem en Actor (bruger) og systemet.



%TODO: Beskriv metoder, og hvordan vi brugte dem. 
% undlad at skrive diskussion om dem her. det h�rer til i Diskussion afsnittet.
