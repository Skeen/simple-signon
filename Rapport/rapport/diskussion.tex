%TODO: Diskussion af valgt l�sning
%TODO: Diskussion af problemer der var undervejs.
%TODO: Diskussion af vores valgte strategi
%TODO: V�lg et element af diskussion ud, og g� i dybten med denne.
% Perspektiver elementet forhold til kursusliteratur.
% (s�t det i forhold til artiklerne fra f�rste 7 uger)

\textbf{\large De valgte l�sninger og hvorfor/alternativer:}


\begin{itemize}
\item Brugerstyret vs. Administrationsstyret
\item Lokalt program vs. Webservice
\item GUI\\
\end{itemize}


Et af de f�rste sp�rgsm�l vi st�dte p� i forbindelse med projektet, var hvorvidt systemet skulle v�re brugerstyret eller administrationsstyret.


Argument mod brugerstyret: \\
kan brugeren fucke det op vil brugeren fucke de om(de kan kun smadre deres eget). \\
Er brugeren i stand til og har kompetance til at administere systemet/mulighederne.\\
Kan tilf�je kompleksitet i udviklingen blandt andet meget vs. lidt GUI/funktionalitet(skal designe til flere brugere).\\

Argumenter for brugerstyret:\\
Den en enkelte bruger kan tilpasse systemet til dem.\\
Hvis der er centralstyring skal alle brugere forbi IT-afdelingen med deres password(alle password skal igennem IT-afdelingen).\\
Mindre arbejde til IT-afdelingen.


 


\leavevmode \linebreak
\textbf{\large Problemer undervejs og de valgte l�sninger:}


\begin{itemize}
\item Proxy som l�sning p� cookie-problem
\item Fleksibilitet
\item Sikkerhed
\item Problemer med 'dummy'-logins\\
\end{itemize}


sd<sd


\leavevmode \linebreak
\textbf{\large Diskussion af vores valgte strategier:}


\begin{itemize}
\item (Cotextual Interviews med observationer)
\item Prototyping/prototype\\
\end{itemize}

sdad


\leavevmode \linebreak
\textbf{\large G� i dybden med �n:}


\begin{itemize}
\item Prototyping/prototype\\
\end{itemize}

asda